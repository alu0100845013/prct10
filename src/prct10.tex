\documentclass[spanish,a4paper,10pt]{article}
\usepackage[spanish]{babel}
\usepackage[utf8]{inputenc}
\usepackage{graphicx}
\usepackage[dvips]{epsfig}
\usepackage{doc}

\begin{document}
\title{Número $\pi$}
\author{Iván Hernández Torres}
\date{11 de abril de 2014}

\maketitle

\begin{abstract}
El objetivo de esta práctica es crear un artículo con el que podamos conocer más información del número PI
\end{abstract}

%\thispagestyle{empty}
%++++++++++++++++++++++++++++++++++++++++++++++++++++++++++++++++++++++
\section{Motivación y Objetivos}

A lo largo de la historia han sido muchas las formas utilizadas por el
ser humano para calcular aproximaciones cada vez más exactas del número $\pi$.
%
Debido a la importancia que tiene el número PI en el ámbito matemático es de gran importancia conocer mejor este número
que en tantas ocaciones utilizaremos a lo largo de nuestro grado y de nuestra vida.
%
%++++++++++++++++++++++++++++++++++++++++++++++++++++++++++++++++++++++
\section{número PI}
\includegraphics[scale=0.15]{imagen1.eps}

\subsection{Definición número Pi}
El número pi, representado por la letra griega $\pi$, equivale a la constante que relaciona el perímetro o longitud de una
circunferencia con su diámetro. Se trata de un valor con un infinito número de decimales, cuya secuencia comienza de
la siguiente manera:3,1415926535897932384626433832795028841
\subsection{Historia del número Pi}
El número pi es la constante que relaciona el perímetro de una circunferencia (L),
con la longitud de su diámetro p = L/D. Este no es un número exacto, sino que es de los llamados números irracionales
que tiene infinitas cifras decimales sin repetición de períodos. Ya en la antigüedad, se insinuó que todos los círculos
conservaban una estrecha dependencia entre el contorno y su radio, pero tan sólo desde el siglo XVII la correlación se
convirtió en un dígito y fue identificado con el nombre "Pi" (de periphereia, denominación que los griegos daban al
perímetro de un círculo). A lo largo de la historia, a este ilustre guarismo se le han asignado diversas cantidades.
Sin embargo, fue en Grecia donde la correspondencia entre el radio y la longitud de una circunferencia comenzó a consolidarse
como uno de los más insignes enigmas a resolver. Un coetáneo de Sócrates, Antiphon, inscribió en el círculo un cuadrado,
luego un octógono e ideó multiplicar la cantidad de lados hasta el momento en que el polígono obtenido se ajustara casi con
la circunferencia. Euclides precisa en sus Elementos, los pasos al límite necesarios e investiga un sistema consistente en
doblar, al igual que Antiphon, el número de lados de los polígonos regulares y en demostrar la convergencia del procedimiento. 

\begin{footnotesize}
 Ejemplo de pie de página
\end{footnotesize}
\begin{tabular}{lrc}
Dígito & Frecuencia & Z-score\\
\hline
0 & 2.935.072 & -0,4709\\
1 & 2.936.516 & 0,3714\\
2 & 2.936.843 & 0,5186\\
3 & 2.935.205 & -0,4891\\
4 & 2.938.787 & 1,7145\\
5 & 2.936.197 & 0,1212\\
6 & 2.935.504 & -0,3051\\
7 & 2.934.083 & -0,1793\\
8 & 2.935.698 & -0,1858\\
9 & 2.936.095 & -0,0584\\
\end{tabular}
%++++++++++++++++++++++++++++++++++++++++++++++++++++++++++++++++++++++
\section{Entregable}
En la tarea habilitada para esta práctica en el Aula Virtual, se subirá
la dirección del repositorio \textit{github} donde se ha almacenado la práctica.

%++++++++++++++++++++++++++++++++++++++++++++++++++++++++++++++++++++++
\section{Referencias bibliográficas}
Para escribir este artículos nos hemos basado en las siguientes fuentes:

\begin{thebibliography}{2}
\bibitem{def} Definición número PI. http://www.saberia.com/2010/03/que-es-el-numero-pi/
\bibitem{hist} Historia número PI. http://www.monografias.com/trabajos57/calculo-pi/calculo-pi.shtml
\bibitem{hist} Historia número PI. http://es.wikipedia.org/wiki/N
\end{thebibliography}

\end{document}